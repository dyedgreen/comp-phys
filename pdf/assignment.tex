\documentclass[10pt, a4paper]{article}

\usepackage[utf8]{inputenc}
\usepackage{amsmath}
\usepackage{amssymb}
\usepackage{graphicx}
\usepackage{hyperref}
\usepackage{cite}
\usepackage{fullpage}

\graphicspath{ {./images/} }

\title{Computational Physics Assignment Answers}
\author{Tilman Roeder}
\date{\today}

\renewcommand\thesection{Question \arabic{section}}
\renewcommand\thesubsection{\thesection{} (\alph{subsection})}

\newcommand{\plot}[3]{\begin{figure}[ht]\centering\includegraphics[width=10cm]{#1}\caption{#2}\label{#3}\end{figure}}

\begin{document}
\maketitle

% Question 1
\section{}

  \subsection{}
  \label{sec:cgo}
  The program can be found in \texttt{/assignment/q-1}. It reports the smallest floating point number
  $a$ such that $1 + a > 1$ as\footnotemark{}:
  \begin{equation}
    a = 2^{-63}.
  \end{equation}

  Note that this is for the C numeric type \texttt{long double}, on a 2017 MacBook Pro. This size
  is the one that would be expected for an 80-bit extended precision floating point (IEEE 754).

  \subsection{}
  Go supports IEEE 32 and 64 bit floating point numbers. For these types we find:
  \begin{itemize}
    \item \texttt{float32} $a \approx 1.192093 \times 10^{-7}$
    \item \texttt{float64} $a \approx 2.220446 \times 10^{-16}$
  \end{itemize}

  Note that Go does not support extended precision floating point values (although they can be used through
  CGo, as is done for \ref{sec:cgo}). These values match the expected values (where `expected' means `values
  they need to be to be IEEE compliant'). The theoretical values given our above definition are:
  \begin{itemize}
    \item \texttt{float32} $a = 2^{-23}$
    \item \texttt{float64} $a = 2^{-52}$
  \end{itemize}

  \footnotetext{Machine $\epsilon$ is also sometimes defined as $\frac{a}{2}$, with a defined as above.
  The values given for $a$ are valid for the definition given above.}

% Question 2
\section{}
  \subsection{}
  The program can be found in \texttt{/assignment/q-2}. The LU decomposition is implemented in the
  \texttt{/assignment/comply} package.

  \subsection{}
  Using the LU decomposition routine, one can find\footnotemark:
  \begin{equation}
    L = \begin{bmatrix}1 & 0 & 0 & 0 & 0 \\ 1 & 1 & 0 & 0 & 0 \\ 0 & 1.125 & 1 & 0 & 0 \\ 0 & 0 & -1.419\cdots & 1 & 0 \\ 0 & 0 & 0 & -1.216\cdots & 1\end{bmatrix}
  \end{equation}
  \begin{equation}
    U = \begin{bmatrix}3 & 1 & 0 & 0 & 0 \\ 0 & 8 & 4 & 0 & 0 \\ 0 & 0 & 15.5 & 10 & 0 \\ 0 & 0 & 0 & 45.193\cdots & -25 \\ 0 & 0 & 0 & 0 & 30.575\cdots\end{bmatrix},
  \end{equation}
  where $LU = A$, with $A$ being the matrix given in the assignment problem.

  \footnotetext{Some of the numeric answers for these questions have very long decimal representations
    when the full result is quoted to machine precision. These numbers have been truncated at three decimal
    figures and are reported as $0.123\cdots$. These merely indicate truncation and no rounding has taken place. To
    see the full results to machine precision, please run \texttt{make assignment}.}

  Further, it can be obtained:
  \begin{equation}
    \det(A) = \det(L) \times \det(U) = \det(U) = \prod_{i=1}^5 U_{ii} = 514032.
  \end{equation}

  \subsection{}
  The solver is implemented in \texttt{/assignment/comply}.

  \subsection{}
  Using the solver, $x$ is determined to be:
  \begin{equation}
    x \approx \begin{bmatrix}0.4565707971488156 \\ 0.6302876085535531 \\ -0.5105752171071062 \\ 0.05389158651601452 \\ 0.19613175833411145\end{bmatrix}.
  \end{equation}

  \subsection{}
  Using the solver, the matrix inverse is determined as:
  \begin{equation}
    A^{-1} \approx \begin{bmatrix}0.379\cdots & -0.046\cdots & 0.004\cdots & -0.004\cdots & -0.001\cdots \\ -0.138\cdots & 0.138\cdots & -0.012\cdots & 0.014\cdots & 0.005\cdots \\ 0.027\cdots & -0.027\cdots & 0.024\cdots & -0.028\cdots & -0.011\cdots \\ 0.070\cdots & -0.070\cdots & 0.062\cdots & 0.044\cdots & 0.018\cdots \\ 0.063\cdots & -0.063\cdots & 0.056\cdots & 0.039\cdots & 0.032\cdots\end{bmatrix}
  \end{equation}

% Question 3
\section{}
  \subsection{}
  Linear interpolation is implemented in \texttt{/pkg/interpolate}.

  \subsection{}
  Cubic spline interpolation is implemented in \texttt{/pkg/interpolate}. However this implementation is
  not fully compliant with the question requirements, so there is an additional (compliant but slower and
  more memory intensive) implementation in \texttt{/assignment/comply}.

  \subsection{}
  The data and interpolations are plotted in figure \ref{fig:interpolate}.

  \plot{assignment-q-3}{
    Linear and natural cubic spline interpolation on the given data.
  }{fig:interpolate}

% Question 4
\section{}
  \subsection{}
  The program can be found in \texttt{/assignment/q-4}. The convolutions are implemented in \texttt{/pkg/signal}.

  The program samples both functions over the range $[-10; 10]$ and takes $2^{10} = 1024$ samples. This
  is an exact power of two, which makes good use of the fast Fourier transform.

  A sample density of $1000$ was chosen for two reasons: firstly, this returns a convolution
  sampled at a rate which results in a smooth-looking graph. Secondly, trials using the Fourier
  transform show that for the chosen range and sample-density, we retrieve the original series very
  precisely with minimal edge-effects or aliasing. This should then lead to good results when
  using the Fourier transform to compute the convolution.

  Our choices are validated, when comparing the numerical result to the exact result, which can
  be obtained by computing the convolution integral directly. (See figure \ref{fig:conv} and equation
  \ref{eq:conv}.)

  \subsection{}
  The plot showing $h(t)$, $g(t)$, and $(g * h)(t)$ can be seen in figure \ref{fig:conv}.

  % Given by: integrate(4 * exp((-(t-tau)**2)/2) / sqrt(2*pi), (tau, 3, 5))
  The theoretical result
  \begin{equation}
    \label{eq:conv}
    (g * h)(t) = - 2 \operatorname{erf}{\left(\frac{\sqrt{2} \left(3 - t\right)}{2} \right)} + 2 \operatorname{erf}{\left(\frac{\sqrt{2} \left(5 - t\right)}{2} \right)}
  \end{equation}
  is also shown in the plot.

  \plot{assignment-q-4}{
    Convolution of the function g and h. The plot depicts the functions themselves, their exact
    convolution, and the result of their numerical convolution.
  }{fig:conv}

% Question 5
\section{}
  \subsection{}
  The resulting distribution can be seen in figure \ref{fig:uniform}. The used random sampling
  is ultimately based on \texttt{PCG XSL RR 128/64}, which is a random number generator with
  very robust statistical properties and small memory requirements\cite{pcg}.

  \plot{assignment-q-5-a}{
    Random samples drawn from a uniform distribution over $[0,1]$. The red line indicates the shape of
    the sample distribution.
  }{fig:uniform}

  \subsection{}
  \label{sec:sample}
  Starting from a uniform variate $x \in [0,1]$ with $x \sim p(x) = 1$, consider:

  \begin{equation}
    \int_0^x d\zeta p_x(\zeta) = \int_0^{y(x)} d\gamma p_y(\gamma) = C_y(y(x)),
  \end{equation}
  from where it follows that
  \begin{equation}
    y(x) = C_y^{-1}(x) = \left(\int_0^y d\gamma p_y(\gamma)\right)^{-1}(x).
  \end{equation}

  We are given $p_y(y) = \frac{1}{2} \cos(\frac{y}{2})$, so then:
  \begin{equation}
    C_y(y) = \int_0^y d\gamma \frac{1}{2} \cos(\frac{\gamma}{2}) = \sin(\frac{y}{2}),
  \end{equation}
  from where we finally find:
  \begin{equation}
    y(x) = C_y^{-1}(x) = 2 \times \arcsin(x).
  \end{equation}

  The resulting sample distribution can be seen in figure \ref{fig:sample}.

  \plot{assignment-q-5-b}{
    Samples of $x \sim \frac{1}{2} \cos(\frac{x}{2})$. The red line indicates the shape of
    the sample distribution.
  }{fig:sample}

  \subsection{}
  We implement a simple rejection method in \texttt{/assignment/comply}. The implementations
  interface is modeled on the rejection sampling interface provided by Gonum.

  We use the distribution from \ref{sec:sample} as the proposal distribution, with a value
  of $c = 1.3$. (This $c$ is chosen to minimize the rejection probability.) The resulting
  sample distribution can be seen in figure \ref{fig:reject}.

  \plot{assignment-q-5-c}{
    Samples of $x \sim \frac{2}{\pi} \cos^2(\frac{x}{2})$. The red line indicates the shape of
    the sample distribution.
  }{fig:reject}

  The ration of the time taken versus \ref{sec:sample} is approximately $1.29$. We know that the
  expected number of samples taken is $\mathbb{E}(N) = n \times c$, where $n$ is the number of samples we
  wish to generate. Note that the relative time taken varies with every run, and depends on additional
  factors that lie outside the program itself. Running the binary multiple times shows that the
  average ration is $\approx 1.3$ as expected.

\bibliography{assignment.bib}{}
\bibliographystyle{plain}

\end{document}
