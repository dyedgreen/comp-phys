\documentclass[10pt, a4paper]{article}

\usepackage[utf8]{inputenc}
\usepackage{amsmath}
\usepackage{graphicx}
\usepackage{hyperref}
\usepackage{cite}

\graphicspath{ {./images/} }

\title{Computational Physics Assignment Answers}
\author{Tilman Roeder}
\date{\today}

\renewcommand\thesection{Question \arabic{section}}
\renewcommand\thesubsection{\thesection{} (\alph{subsection})}

\begin{document}
\maketitle

% Question 1
\section{}

  \subsection{}
  The program can be found in \texttt{/assignment/q-1}. It reports the smallest floating point number
  $a$ such that $1 + a > 1$ as\footnotemark{}:
  \begin{equation}
    a = 2^{-63}.
  \end{equation}

  Note that this is for the C numeric type \texttt{long double}, on a 2017 MacBook Pro. This size
  is the one that would be expected for an 80-bit extended precision floating point (IEEE 754).

  \subsection{}
  Go supports IEEE 32 and 64 bit floating point numbers. For these types we find:
  \begin{itemize}
    \item \texttt{float32} $a \approx 1.192093 \times 10^{-7}$
    \item \texttt{float64} $a \approx 2.220446 \times 10^{-16}$
  \end{itemize}

  Note that Go does not support extended precision floating point values (although they can be used through
  CGo, as is done for question (a)). These values match the expected values (where `expected' means `values
  they need to be to be IEEE compliant').

  \footnotetext{Machine $\epsilon$ is also sometimes defined as $\frac{a}{2}$, with a defined as above.
  The values given for $a$ are valid for the definition given above.}

% Question 2
\section{}
  \subsection{}
  The program can be found in \texttt{/assignment/q-2}. The LU decomposition is implemented in the
  \texttt{/assignment/comply} package.

  \subsection{}
  Using the LU decomposition routine, one can find found\footnotemark:
  \begin{equation}
    L = \begin{bmatrix}1 & 0 & 0 & 0 & 0 \\ 1 & 1 & 0 & 0 & 0 \\ 0 & 1.125 & 1 & 0 & 0 \\ 0 & 0 & -1.419\cdots & 1 & 0 \\ 0 & 0 & 0 & -1.216\cdots & 1\end{bmatrix}
  \end{equation}
  \begin{equation}
    U = \begin{bmatrix}3 & 1 & 0 & 0 & 0 \\ 0 & 8 & 4 & 0 & 0 \\ 0 & 0 & 15.5 & 10 & 0 \\ 0 & 0 & 0 & 45.193\cdots & -25 \\ 0 & 0 & 0 & 0 & 30.575\cdots\end{bmatrix}.
  \end{equation}
  where $LU = A$, with $A$ being the matrix given in the assignment problem.

  \footnotetext{Some of the numeric answers for these questions have very long decimal representations
    when the full result is quoted to machine precision. These numbers have been truncated at three decimal
    figures and are reported as $0.123\cdots$. These merely indicate truncation and no rounding has taken place. To
    see the full results to machine precision, please run \texttt{make assignment}.}

  Further, it can be obtained:
  \begin{equation}
    \det(A) = \det(L) \times \det(U) = \det(U) = \prod_{i=1}^5 U_{ii} = 514032.
  \end{equation}

  \subsection{}
  The solver is implemented in \texttt{/assignment/comply}.

  \subsection{}
  Using the solver, $x$ is determined to be:
  \begin{equation}
    x \approx \begin{bmatrix}0.4565707971488156 \\ 0.6302876085535531 \\ -0.5105752171071062 \\ 0.05389158651601452 \\ 0.19613175833411145\end{bmatrix}.
  \end{equation}

  \subsection{}
  Using the solver, the matrix inverse is determined as:
  \begin{equation}
    A^{-1} \approx \begin{bmatrix}0.379\cdots & -0.046\cdots & 0.004\cdots & -0.004\cdots & -0.001\cdots \\ -0.138\cdots & 0.138\cdots & -0.012\cdots & 0.014\cdots & 0.005\cdots \\ 0.027\cdots & -0.027\cdots & 0.024\cdots & -0.028\cdots & -0.011\cdots \\ 0.070\cdots & -0.070\cdots & 0.062\cdots & 0.044\cdots & 0.018\cdots \\ 0.063\cdots & -0.063\cdots & 0.056\cdots & 0.039\cdots & 0.032\cdots\end{bmatrix}
  \end{equation}

% Question 3
\section{}
  \subsection{}
  Linear interpolation is implemented in \texttt{/pkg/interpolate}.

  \subsection{}
  Cubic spline interpolation is implemented in \texttt{/pkg/interpolate}. However this implementation is
  not fully compliant with the question requirements, so there is an additional (compliant but slower and
  more memory intensive) implementation in \texttt{/assignment/comply}.

  \subsection{}
  The data and interpolations are plotted in figure \ref{fig:interpolate}.

  \begin{figure}[h]
    \centering
    \includegraphics[width=12cm]{assignment-q-3}
    \caption{Linear and natural cubic spline interpolation on the given data.}
    \label{fig:interpolate}
  \end{figure}

% Question 4
\section{}
  \subsection{}

% Question 5
\section{}
  \subsection{}
  The resulting distribution can be seen in figure \ref{fig:uniform}. The used random sampling
  is ultimately based on \texttt{PCG XSL RR 128/64}, which is a random number generator with
  very robust statistical properties and small memory requirements\cite{pcg}.

  \begin{figure}[h]
    \centering
    \includegraphics[width=12cm]{assignment-q-5-1}
    \caption{Random samples drawn from a uniform distribution over $[0,1]$.}
    \label{fig:uniform}
  \end{figure}

\bibliography{assignment.bib}{}
\bibliographystyle{plain}

\end{document}
