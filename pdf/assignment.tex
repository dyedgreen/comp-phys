\documentclass[12pt, a4paper]{article}

\usepackage[utf8]{inputenc}
\usepackage{amsmath}

\title{Computational Physics Assignment Answers}
\author{Tilman Roeder}
\date{\today}

\renewcommand\thesection{Question \arabic{section}}
\renewcommand\thesubsection{\thesection{} (\alph{subsection})}

\begin{document}
\maketitle

\section{}
  \subsection{}
  The program can be found in \texttt{/assignment/q-1}. It reports the smallest floating point number
  $a$ such that $1 + a > 1$ as\footnotemark{}:
  \begin{equation}
    a = 2^{-63}.
  \end{equation}

  Note that this is for the C numeric type \texttt{long double}, on a 2017 MacBook Pro. This size
  is the one that would be expected for an 80-bit extended precision floating point (IEEE 754).

  \subsection{}
  Go supports IEEE 32 and 64 bit floating point numbers. For these types we find:
  \begin{itemize}
    \item \texttt{float32} $a \approx 1.192093 \times 10^{-7}$
    \item \texttt{float64} $a \approx 2.220446 \times 10^{-16}$
  \end{itemize}

  Note that Go does not support extended precision floating point values (although they can be used through
  CGo, as is done for question (a)). These values match the expected values (where `expected' means `values
  they need to be to be IEEE compliant').

  \footnotetext{Machine $\epsilon$ is also sometimes defined as $\frac{a}{2}$, with a defined as above.
  The values given for $a$ are valid for the definition given above.}

\section{}
  \subsection{}
  The program can be found in \texttt{/assignment/q-2}. The LU decomposition is implemented in the
  \texttt{/assignment/comply} package.

  \subsection{}
  Using the LU decomposition routine, one can find found:
  \begin{equation}
    L = \begin{bmatrix}1 & 0 & 0 & 0 & 0 \\ 1 & 1 & 0 & 0 & 0 \\ 0 & 1.125 & 1 & 0 & 0 \\ 0 & 0 & -1.419\cdots & 1 & 0 \\ 0 & 0 & 0 & -1.216\cdots & 1\end{bmatrix}
  \end{equation}
  \begin{equation}
    U = \begin{bmatrix}3 & 1 & 0 & 0 & 0 \\ 0 & 8 & 4 & 0 & 0 \\ 0 & 0 & 15.5 & 10 & 0 \\ 0 & 0 & 0 & 45.193\cdots & -25 \\ 0 & 0 & 0 & 0 & 30.575\cdots\end{bmatrix}.
  \end{equation}
  where $LU = A$, with $A$ being the matrix given in the assignment problem.

  Further, it can be obtained:
  \begin{equation}
    \det(A) = \det(L) \times \det(U) = \det(U) = \prod_{i=1}^5 U_{ii} = 514032.
  \end{equation}

  \subsection{}
  The solver is implemented in \texttt{/assignment/comply}.


\end{document}
