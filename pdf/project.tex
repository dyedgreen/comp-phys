\documentclass[10pt, a4paper]{article}

\usepackage[utf8]{inputenc}
\usepackage{amsmath}
\usepackage{amssymb}
\usepackage{graphicx}
\usepackage{hyperref}
\usepackage{cite}
\usepackage{fullpage}

\graphicspath{ {./images/} }

\title{Computational Physics Project \\ Integrating Quantum Probability Densities}
\author{Tilman Roeder}
\date{\today}

% Uniform plots
\newcommand{\plot}[3]{\begin{figure}[ht]\centering\includegraphics[width=10cm]{#1}\caption{#2}\label{#3}\end{figure}}

% Abbreviations
\newcommand{\abbreviate}[3]{\newcommand{#1}{#3 \textit{(#2)}\renewcommand{#1}{#2}}}
\abbreviate{\apis}{APIS}{Adaptive Population Importance Sampling}


\begin{document}
\maketitle

\section{Introduction}
In this report we investigate methods for numerically computing the value of a proper integral.
The relevant integral under scrutiny is

\begin{equation}
\label{eq:target}
I = \int_a^b |\Psi(x)|^2 dx = \int_a^b \frac{1}{\sqrt{\pi}} e^{-x^2} dx,
\end{equation}
where we are interested in $a = 0$ and $b = 2$.

This is, of course, the normalized Gaussian probability density, for which
very accurate approximations and tabulated values exist. However, for the purposes of this report, we shall
treat the value of the integral $I$ as unknown and investigate how to justify our confidence in the
obtained numerical results.

A comparison of our results with the tabulated values for this integral can be found in appendix
\ref{app:cheat}.

\section{Algorithms}
We will explore two general classes of algorithms: deterministic quadrature methods, and different
Monte-Carlo integration schemes.

\subsection{Quadrature Methods}
  \subsubsection{Trapezoidal Quadrature}

  \subsubsection{Simpson Rule}

\subsection{Monte-Carlo Integration}
  \subsubsection{Importance Sampling}
  When considering any integral, we may notice that
  \begin{equation}
  \int_a^b f(x) dx = \int_{a}^{b} \frac{f(x)}{p(x)} p(x) dx = \mathbb{E}_{\sim p(x)}\left[ \frac{f(x)}{p(x)} \right],
  \end{equation}
  where $p(x)$ is a probability measure with finite support $[a, b]$ and $\mathbb{E}$ denotes the expected
  value.

  TODO - Explain the implementation of expected value (and online mean, variance)
  TODO - explain random number generator used + reference paper
  TODO - explain set of random seeds used and where they originate from
  TODO - mention parallel runtime implementation

- explain seeding and maybe change it to supply a range of seeds
  - maybe allow users to add a 'seed-bank'

  \subsubsection{Adaptive Population Importance Sampling}
  explain the \apis{} algorithm briefly, and reference paper

\section{Ensuring Implementation Correctness}

- explain unit tests
- outline tests run and test coverage

\section{Results}

- present results

- mention that $10^{-6}$ is impossible with MC, report average time for 1 function evaluation given
  full process parallelization and use this to calculate the order of magnitude of time taken to
  get to $10^{-6}$, also mention that this will be much more favorable for higher-dimensional integrals

\section{Verification of Results}

- show that integrand is smooth (possibly even Lipschitz), and proof/ explain why this leads to
  correct results in integral from quadrature method
  - analyze numeric error under perfect arithmetic for our integrand

- analyze numeric properties of quadrature method used and estimate numeric error incurred

- analyze numeric stability of monte-carlo method used

- outline variance on integral supplied and explain that we use a two-sigma error range to
  and how we expect that to behave

- compare results of different algorithms + show they agree
- have graphs of how integral values change w/ iterations
  - including errors for monte-carlo code
  - include theoretical "bounds", starting from the converged value

\bibliography{assignment.bib}{}
\bibliographystyle{plain}

\appendix{}

\section{Comparison with Tabulated Values}
\label{app:cheat}
From tables we obtain
\begin{equation}
I = \frac{\operatorname{erf}(2)}{2} = 0.49766113250947636708 \dots,
\end{equation}
where $\operatorname{erf}(x) = \frac{1}{\sqrt\pi}\int_{-x}^x e^{-t^2} dt$ is the Error function.

TODO: Compare result with tabulated values here ...

\end{document}
