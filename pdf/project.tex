\documentclass[10pt, a4paper]{article}

\usepackage[utf8]{inputenc}
\usepackage{amsmath}
\usepackage{amssymb}
\usepackage{graphicx}
\usepackage{hyperref}
\usepackage{cite}
\usepackage{fullpage}

\graphicspath{ {./images/} }

\title{Computational Physics Project}
\author{Tilman Roeder}
\date{\today}

\newcommand{\plot}[3]{\begin{figure}[ht]\centering\includegraphics[width=10cm]{#1}\caption{#2}\label{#3}\end{figure}}

\begin{document}
\maketitle

\section{Plan}
- Quad package -> standard integration stuff
- Casino package -> monte carlo routines
  -> used by some of the methods in quad (or maybe have some specialized methods in casino)
- Extensions (no: focus on validation ...)
  -> use numerov to solve QM / Schrödinger Equation
  -> use monte carlo to solve PDE (Schrödinger Equation)

\section{Introduction}

- introduce problem

\section{Algorithms}

- explain seeding and maybe change it to supply a range of seeds
  - maybe allow users to add a 'seed-bank'

\section{Ensuring Implementation Correctness}

- explain unit tests
- outline tests run and test coverage

\section{Results}

- present results
- have graphs of how integral values change w/ iterations
  - including errors for monte-carlo code

\section{Verification of Results}

- show that integrand is smooth (possibly even Lipschitz), and proof/ explain why this leads to
  correct results in integral from quadrature method
  - analyze numeric error under perfect arithmetic for our integrand

- analyze numeric properties of quadrature method used and estimate numeric error incurred

- analyze numeric stability of monte-carlo method used

- outline variance on integral supplied and explain that we use a two-sigma error range to
  and how we expect that to behave

- compare results of different algorithms + show they agree

\section{Approximate Transform For Normal Distribution}
\begin{equation}
\sqrt{2} \sigma \erf^{-1}(2x - 1) + \mu = y
\end{equation}

\bibliography{assignment.bib}{}
\bibliographystyle{plain}

\appendix{}

\section{Appendix here}
Bla blub.

\end{document}
